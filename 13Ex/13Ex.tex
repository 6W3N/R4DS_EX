\documentclass[]{ltjsarticle}
\usepackage{lmodern}
\usepackage{amssymb,amsmath}
\usepackage{ifxetex,ifluatex}
\usepackage{fixltx2e} % provides \textsubscript
\ifnum 0\ifxetex 1\fi\ifluatex 1\fi=0 % if pdftex
  \usepackage[T1]{fontenc}
  \usepackage[utf8]{inputenc}
\else % if luatex or xelatex
  \ifxetex
    \usepackage{mathspec}
  \else
    \usepackage{fontspec}
  \fi
  \defaultfontfeatures{Ligatures=TeX,Scale=MatchLowercase}
\fi
% use upquote if available, for straight quotes in verbatim environments
\IfFileExists{upquote.sty}{\usepackage{upquote}}{}
% use microtype if available
\IfFileExists{microtype.sty}{%
\usepackage{microtype}
\UseMicrotypeSet[protrusion]{basicmath} % disable protrusion for tt fonts
}{}
\usepackage[margin=1in]{geometry}
\usepackage{hyperref}
\hypersetup{unicode=true,
            pdftitle={r4ds Ex 13},
            pdfauthor={MW},
            pdfborder={0 0 0},
            breaklinks=true}
\urlstyle{same}  % don't use monospace font for urls
\usepackage{color}
\usepackage{fancyvrb}
\newcommand{\VerbBar}{|}
\newcommand{\VERB}{\Verb[commandchars=\\\{\}]}
\DefineVerbatimEnvironment{Highlighting}{Verbatim}{commandchars=\\\{\}}
% Add ',fontsize=\small' for more characters per line
\usepackage{framed}
\definecolor{shadecolor}{RGB}{248,248,248}
\newenvironment{Shaded}{\begin{snugshade}}{\end{snugshade}}
\newcommand{\AlertTok}[1]{\textcolor[rgb]{0.94,0.16,0.16}{#1}}
\newcommand{\AnnotationTok}[1]{\textcolor[rgb]{0.56,0.35,0.01}{\textbf{\textit{#1}}}}
\newcommand{\AttributeTok}[1]{\textcolor[rgb]{0.77,0.63,0.00}{#1}}
\newcommand{\BaseNTok}[1]{\textcolor[rgb]{0.00,0.00,0.81}{#1}}
\newcommand{\BuiltInTok}[1]{#1}
\newcommand{\CharTok}[1]{\textcolor[rgb]{0.31,0.60,0.02}{#1}}
\newcommand{\CommentTok}[1]{\textcolor[rgb]{0.56,0.35,0.01}{\textit{#1}}}
\newcommand{\CommentVarTok}[1]{\textcolor[rgb]{0.56,0.35,0.01}{\textbf{\textit{#1}}}}
\newcommand{\ConstantTok}[1]{\textcolor[rgb]{0.00,0.00,0.00}{#1}}
\newcommand{\ControlFlowTok}[1]{\textcolor[rgb]{0.13,0.29,0.53}{\textbf{#1}}}
\newcommand{\DataTypeTok}[1]{\textcolor[rgb]{0.13,0.29,0.53}{#1}}
\newcommand{\DecValTok}[1]{\textcolor[rgb]{0.00,0.00,0.81}{#1}}
\newcommand{\DocumentationTok}[1]{\textcolor[rgb]{0.56,0.35,0.01}{\textbf{\textit{#1}}}}
\newcommand{\ErrorTok}[1]{\textcolor[rgb]{0.64,0.00,0.00}{\textbf{#1}}}
\newcommand{\ExtensionTok}[1]{#1}
\newcommand{\FloatTok}[1]{\textcolor[rgb]{0.00,0.00,0.81}{#1}}
\newcommand{\FunctionTok}[1]{\textcolor[rgb]{0.00,0.00,0.00}{#1}}
\newcommand{\ImportTok}[1]{#1}
\newcommand{\InformationTok}[1]{\textcolor[rgb]{0.56,0.35,0.01}{\textbf{\textit{#1}}}}
\newcommand{\KeywordTok}[1]{\textcolor[rgb]{0.13,0.29,0.53}{\textbf{#1}}}
\newcommand{\NormalTok}[1]{#1}
\newcommand{\OperatorTok}[1]{\textcolor[rgb]{0.81,0.36,0.00}{\textbf{#1}}}
\newcommand{\OtherTok}[1]{\textcolor[rgb]{0.56,0.35,0.01}{#1}}
\newcommand{\PreprocessorTok}[1]{\textcolor[rgb]{0.56,0.35,0.01}{\textit{#1}}}
\newcommand{\RegionMarkerTok}[1]{#1}
\newcommand{\SpecialCharTok}[1]{\textcolor[rgb]{0.00,0.00,0.00}{#1}}
\newcommand{\SpecialStringTok}[1]{\textcolor[rgb]{0.31,0.60,0.02}{#1}}
\newcommand{\StringTok}[1]{\textcolor[rgb]{0.31,0.60,0.02}{#1}}
\newcommand{\VariableTok}[1]{\textcolor[rgb]{0.00,0.00,0.00}{#1}}
\newcommand{\VerbatimStringTok}[1]{\textcolor[rgb]{0.31,0.60,0.02}{#1}}
\newcommand{\WarningTok}[1]{\textcolor[rgb]{0.56,0.35,0.01}{\textbf{\textit{#1}}}}
\usepackage{graphicx,grffile}
\makeatletter
\def\maxwidth{\ifdim\Gin@nat@width>\linewidth\linewidth\else\Gin@nat@width\fi}
\def\maxheight{\ifdim\Gin@nat@height>\textheight\textheight\else\Gin@nat@height\fi}
\makeatother
% Scale images if necessary, so that they will not overflow the page
% margins by default, and it is still possible to overwrite the defaults
% using explicit options in \includegraphics[width, height, ...]{}
\setkeys{Gin}{width=\maxwidth,height=\maxheight,keepaspectratio}
\IfFileExists{parskip.sty}{%
\usepackage{parskip}
}{% else
\setlength{\parindent}{0pt}
\setlength{\parskip}{6pt plus 2pt minus 1pt}
}
\setlength{\emergencystretch}{3em}  % prevent overfull lines
\providecommand{\tightlist}{%
  \setlength{\itemsep}{0pt}\setlength{\parskip}{0pt}}
\setcounter{secnumdepth}{0}
% Redefines (sub)paragraphs to behave more like sections
\ifx\paragraph\undefined\else
\let\oldparagraph\paragraph
\renewcommand{\paragraph}[1]{\oldparagraph{#1}\mbox{}}
\fi
\ifx\subparagraph\undefined\else
\let\oldsubparagraph\subparagraph
\renewcommand{\subparagraph}[1]{\oldsubparagraph{#1}\mbox{}}
\fi

%%% Use protect on footnotes to avoid problems with footnotes in titles
\let\rmarkdownfootnote\footnote%
\def\footnote{\protect\rmarkdownfootnote}

%%% Change title format to be more compact
\usepackage{titling}

% Create subtitle command for use in maketitle
\newcommand{\subtitle}[1]{
  \posttitle{
    \begin{center}\large#1\end{center}
    }
}

\setlength{\droptitle}{-2em}

  \title{r4ds Ex 13}
    \pretitle{\vspace{\droptitle}\centering\huge}
  \posttitle{\par}
    \author{MW}
    \preauthor{\centering\large\emph}
  \postauthor{\par}
      \predate{\centering\large\emph}
  \postdate{\par}
    \date{2019/07/03}


\begin{document}
\maketitle

\hypertarget{section}{%
\section{13.1}\label{section}}

\hypertarget{section-1}{%
\subsection{1}\label{section-1}}

\begin{quote}
Imagine you wanted to draw (approximately) the route each plane flies
from its origin to its destination. What variables would you need? What
tables would you need to combine?
\end{quote}

We need to combine \texttt{flights} and \texttt{airport} due to
requiring latitude and longitude of airports of destination.

\hypertarget{section-2}{%
\subsection{2}\label{section-2}}

\begin{quote}
I forgot to draw the relationship between \texttt{weather} and
\texttt{airports}. What is the relationship and how should it appear in
the diagram?
\end{quote}

\texttt{airports\$faa} and \texttt{weather\$origin} are relationships as
foreign keys.

\hypertarget{section-3}{%
\subsection{3}\label{section-3}}

\begin{quote}
\texttt{weather} only contains information for the origin (NYC)
airports. If it contained weather records for all airports in the USA,
what additional relation would it define with \texttt{flights}?
\end{quote}

There are weather of destinations.

\hypertarget{section-4}{%
\subsection{4}\label{section-4}}

\begin{quote}
We know that some days of the year are ``special'', and fewer people
than usual fly on them. How might you represent that data as a data
frame? What would be the primary keys of that table? How would it
connect to the existing tables?
\end{quote}

\begin{Shaded}
\begin{Highlighting}[]
\NormalTok{holiday <-}\StringTok{ }\KeywordTok{tribble}\NormalTok{(}
    \OperatorTok{~}\NormalTok{year, }\OperatorTok{~}\NormalTok{month, }\OperatorTok{~}\NormalTok{day, }\OperatorTok{~}\NormalTok{holiday,}
    \DecValTok{2013}\NormalTok{, }\DecValTok{01}\NormalTok{, }\DecValTok{01}\NormalTok{, }\StringTok{"New Year"}\NormalTok{,}
    \DecValTok{2013}\NormalTok{, }\DecValTok{12}\NormalTok{, }\DecValTok{25}\NormalTok{, }\StringTok{"Christmas Day"}
\NormalTok{)}
\end{Highlighting}
\end{Shaded}

\hypertarget{section-5}{%
\section{13.3}\label{section-5}}

\hypertarget{section-6}{%
\subsection{1}\label{section-6}}

\begin{quote}
Add a surrogate key to flights.
\end{quote}

\begin{Shaded}
\begin{Highlighting}[]
\NormalTok{flights }\OperatorTok\StringTok{ }\KeywordTok{mutate}\NormalTok{(}\DataTypeTok{id=}\DecValTok{1}\OperatorTok{:}\KeywordTok{nrow}\NormalTok{(.))}
\end{Highlighting}
\end{Shaded}

\begin{verbatim}
## # A tibble: 336,776 x 20
##     year month   day dep_time sched_dep_time dep_delay arr_time
##    <int> <int> <int>    <int>          <int>     <dbl>    <int>
##  1  2013     1     1      517            515         2      830
##  2  2013     1     1      533            529         4      850
##  3  2013     1     1      542            540         2      923
##  4  2013     1     1      544            545        -1     1004
##  5  2013     1     1      554            600        -6      812
##  6  2013     1     1      554            558        -4      740
##  7  2013     1     1      555            600        -5      913
##  8  2013     1     1      557            600        -3      709
##  9  2013     1     1      557            600        -3      838
## 10  2013     1     1      558            600        -2      753
## # ... with 336,766 more rows, and 13 more variables: sched_arr_time <int>,
## #   arr_delay <dbl>, carrier <chr>, flight <int>, tailnum <chr>,
## #   origin <chr>, dest <chr>, air_time <dbl>, distance <dbl>, hour <dbl>,
## #   minute <dbl>, time_hour <dttm>, id <int>
\end{verbatim}

\hypertarget{section-7}{%
\subsection{2}\label{section-7}}

\begin{quote}
Identify the keys in the following datasets \textgreater{} 1.
Lahman::Batting, \textgreater{} 2. babynames::babynames \textgreater{}
3. nasaweather::atmos \textgreater{} 4. fueleconomy::vehicles
\textgreater{} 5. ggplot2::diamonds (You might need to install some
packages and read some documentation.)
\end{quote}

\hypertarget{section-8}{%
\subsubsection{1.}\label{section-8}}

In 1.-4. there are primary keys of each dataset.

\begin{Shaded}
\begin{Highlighting}[]
\NormalTok{Lahman}\OperatorTok{::}\NormalTok{Batting }\OperatorTok\StringTok{ }\KeywordTok{as_tibble}\NormalTok{() }\OperatorTok
\StringTok{    }\KeywordTok{group_by}\NormalTok{(playerID, yearID, stint) }\OperatorTok\StringTok{ }\CommentTok{#grouping by primary keys}
\StringTok{    }\KeywordTok{mutate}\NormalTok{(}\DataTypeTok{count=}\KeywordTok{n}\NormalTok{()) }\OperatorTok
\StringTok{    }\KeywordTok{filter}\NormalTok{(count}\OperatorTok{>}\DecValTok{1}\NormalTok{)}
\end{Highlighting}
\end{Shaded}

\begin{verbatim}
## # A tibble: 0 x 23
## # Groups:   playerID, yearID, stint [0]
## # ... with 23 variables: playerID <chr>, yearID <int>, stint <int>,
## #   teamID <fct>, lgID <fct>, G <int>, AB <int>, R <int>, H <int>,
## #   X2B <int>, X3B <int>, HR <int>, RBI <int>, SB <int>, CS <int>,
## #   BB <int>, SO <int>, IBB <int>, HBP <int>, SH <int>, SF <int>,
## #   GIDP <int>, count <int>
\end{verbatim}

Basically, 2.-4. are same as 1.

\hypertarget{section-9}{%
\subsubsection{5}\label{section-9}}

There are no primary keys.

\begin{Shaded}
\begin{Highlighting}[]
\NormalTok{ggplot2}\OperatorTok{::}\NormalTok{diamonds }\OperatorTok
\StringTok{    }\KeywordTok{distinct}\NormalTok{() }\OperatorTok
\StringTok{    }\KeywordTok{nrow}\NormalTok{()}
\end{Highlighting}
\end{Shaded}

\begin{verbatim}
## [1] 53794
\end{verbatim}

\begin{Shaded}
\begin{Highlighting}[]
\NormalTok{ggplot2}\OperatorTok{::}\NormalTok{diamonds }\OperatorTok
\StringTok{    }\KeywordTok{nrow}\NormalTok{()}
\end{Highlighting}
\end{Shaded}

\begin{verbatim}
## [1] 53940
\end{verbatim}

\hypertarget{section-10}{%
\subsection{3}\label{section-10}}

\begin{quote}
Draw a diagram illustrating the connections between the
\texttt{Batting}, \texttt{Master}, and \texttt{Salaries} tables in the
Lahman package. Draw another diagram that shows the relationship between
\texttt{Master}, \texttt{Managers}, \texttt{AwardsManagers}. How would
you characterise the relationship between the \texttt{Batting},
\texttt{Pitching}, and \texttt{Fielding} tables?
\end{quote}

We can create diagram in R using \texttt{DiagrameR} or
\texttt{datamodelr}.

\begin{Shaded}
\begin{Highlighting}[]
\NormalTok{DiagrammeR}\OperatorTok{::}\KeywordTok{grViz}\NormalTok{(}\StringTok{"13Ex/1.dot"}\NormalTok{)}
\NormalTok{DiagrammeR}\OperatorTok{::}\KeywordTok{grViz}\NormalTok{(}\StringTok{"13Ex/2.dot"}\NormalTok{)}
\end{Highlighting}
\end{Shaded}

1.dot

\begin{verbatim}
digraph subgraph_label {
  rankdir = TB
  subgraph cluster0{
    yearID_S[label="yearID"]
    teamID_S[label="teamID"]
    playerID_S[label="playerID"]
    label = "Salaries"
    {rank = same; yearID; teamID; playerID;}
  }
  subgraph cluster1{
    playerID_M[label="playerID"]
    label = "Master"
  }
  subgraph cluster2{
    playerID_B[label="playerID"]
    yearID_B[label="yearID"]
    stint_B[label="stint"]
    label = "Batting"
    {rank = same; yearID; stint; playerID;}
  }
  playerID_M -> playerID_S
  playerID_M -> playerID_B
}
\end{verbatim}

2.dot

\begin{verbatim}
digraph subgraph_label {
  rankdir = TB
  subgraph cluster0{
    yearID_MN[label="yearID"]
    teamID_MN[label="teamID"]
    playerID_MN[label="playerID"]
    inseason_MN[label="inseason"]
    label = "Managers"
    {rank = same; yearID_MN; teamID_MN; inseason_MN; playerID_MN}
  }
  subgraph cluster1{
    playerID_M[label="playerID"]
    label = "Master"
  }
  subgraph cluster2{
    playerID_A[label="playerID"]
    awardID_A[label="awardID"]
    yearID_A[label="yearID"]
    lgID_A[label="lgID"]
    label = "AwardManagers"
    {rank = same; yearID; stint; playerID;}
  }
  playerID_M -> playerID_A
  playerID_M -> playerID_MN
}
\end{verbatim}

\hypertarget{section-11}{%
\section{13.4}\label{section-11}}

\hypertarget{section-12}{%
\subsection{1}\label{section-12}}

\begin{quote}
Compute the average delay by destination, then join on the airports data
frame so you can show the spatial distribution of delays. Here's an easy
way to draw a map of the United States:
\end{quote}

\begin{Shaded}
\begin{Highlighting}[]
\NormalTok{flights }\OperatorTok
\StringTok{    }\KeywordTok{group_by}\NormalTok{(dest) }\OperatorTok
\StringTok{    }\KeywordTok{summarise}\NormalTok{(}\DataTypeTok{delay =} \KeywordTok{mean}\NormalTok{(arr_delay, }\DataTypeTok{na.rm =} \OtherTok{TRUE}\NormalTok{)) }\OperatorTok
\StringTok{    }\KeywordTok{inner_join}\NormalTok{(airports, }\DataTypeTok{by =} \KeywordTok{c}\NormalTok{(}\DataTypeTok{dest =} \StringTok{"faa"}\NormalTok{)) }\OperatorTok
\StringTok{    }\KeywordTok{ggplot}\NormalTok{(}\KeywordTok{aes}\NormalTok{(lon, lat, }\DataTypeTok{colour =}\NormalTok{ delay)) }\OperatorTok{+}
\StringTok{    }\KeywordTok{borders}\NormalTok{(}\StringTok{"state"}\NormalTok{) }\OperatorTok{+}
\StringTok{    }\KeywordTok{geom_point}\NormalTok{() }\OperatorTok{+}
\StringTok{    }\KeywordTok{coord_quickmap}\NormalTok{()}
\end{Highlighting}
\end{Shaded}

\includegraphics{13Ex_files/figure-latex/unnamed-chunk-6-1.pdf}

\hypertarget{section-13}{%
\subsection{2}\label{section-13}}

\begin{quote}
Add the location of the origin and destination (i.e.~the \texttt{lat}
and \texttt{lon}) to \texttt{flights}.
\end{quote}

\begin{Shaded}
\begin{Highlighting}[]
\NormalTok{flights }\OperatorTok\StringTok{ }\KeywordTok{select}\NormalTok{(year}\OperatorTok{:}\NormalTok{day, hour, origin, dest) }\OperatorTok
\StringTok{    }\KeywordTok{left_join}\NormalTok{(}
\NormalTok{          airports,}
          \DataTypeTok{by =} \KeywordTok{c}\NormalTok{(}\StringTok{"origin"}\NormalTok{ =}\StringTok{ "faa"}\NormalTok{)}
\NormalTok{          ) }\OperatorTok
\StringTok{    }\KeywordTok{left_join}\NormalTok{(}
\NormalTok{          airports,}
          \DataTypeTok{by =} \KeywordTok{c}\NormalTok{(}\StringTok{"dest"}\NormalTok{ =}\StringTok{ "faa"}\NormalTok{)}
\NormalTok{    ) }\OperatorTok
\StringTok{    }\KeywordTok{select}\NormalTok{(year}\OperatorTok{:}\NormalTok{day, origin, dest, lat.x, lon.x, lat.y, lon.y)}
\end{Highlighting}
\end{Shaded}

\begin{verbatim}
## # A tibble: 336,776 x 9
##     year month   day origin dest  lat.x lon.x lat.y lon.y
##    <int> <int> <int> <chr>  <chr> <dbl> <dbl> <dbl> <dbl>
##  1  2013     1     1 EWR    IAH    40.7 -74.2  30.0 -95.3
##  2  2013     1     1 LGA    IAH    40.8 -73.9  30.0 -95.3
##  3  2013     1     1 JFK    MIA    40.6 -73.8  25.8 -80.3
##  4  2013     1     1 JFK    BQN    40.6 -73.8  NA    NA  
##  5  2013     1     1 LGA    ATL    40.8 -73.9  33.6 -84.4
##  6  2013     1     1 EWR    ORD    40.7 -74.2  42.0 -87.9
##  7  2013     1     1 EWR    FLL    40.7 -74.2  26.1 -80.2
##  8  2013     1     1 LGA    IAD    40.8 -73.9  38.9 -77.5
##  9  2013     1     1 JFK    MCO    40.6 -73.8  28.4 -81.3
## 10  2013     1     1 LGA    ORD    40.8 -73.9  42.0 -87.9
## # ... with 336,766 more rows
\end{verbatim}

\hypertarget{section-14}{%
\subsection{3}\label{section-14}}

\begin{quote}
Is there a relationship between the age of a plane and its delays?
\end{quote}

\begin{Shaded}
\begin{Highlighting}[]
\NormalTok{flights }\OperatorTok\StringTok{ }\KeywordTok{inner_join}\NormalTok{(planes }\OperatorTok\StringTok{ }\KeywordTok{select}\NormalTok{(tailnum, }\DataTypeTok{plane_year =}\NormalTok{ year), }\DataTypeTok{by =} \StringTok{"tailnum"}\NormalTok{) }\OperatorTok
\StringTok{    }\KeywordTok{mutate}\NormalTok{(}\DataTypeTok{age =}\NormalTok{ year }\OperatorTok{-}\StringTok{ }\NormalTok{plane_year) }\OperatorTok
\StringTok{    }\KeywordTok{group_by}\NormalTok{(age) }\OperatorTok
\StringTok{    }\KeywordTok{summarise}\NormalTok{(}
          \DataTypeTok{dep_delay_mean =} \KeywordTok{mean}\NormalTok{(dep_delay, }\DataTypeTok{na.rm =} \OtherTok{TRUE}\NormalTok{),}
          \DataTypeTok{arr_delay_mean =} \KeywordTok{mean}\NormalTok{(arr_delay, }\DataTypeTok{na.rm =} \OtherTok{TRUE}\NormalTok{),}
          \DataTypeTok{n_arr_delay =} \KeywordTok{sum}\NormalTok{(}\OperatorTok{!}\KeywordTok{is.na}\NormalTok{(arr_delay)),}
          \DataTypeTok{n_dep_delay =} \KeywordTok{sum}\NormalTok{(}\OperatorTok{!}\KeywordTok{is.na}\NormalTok{(arr_delay))}
\NormalTok{          ) }\OperatorTok
\StringTok{    }\KeywordTok{ggplot}\NormalTok{(}\KeywordTok{aes}\NormalTok{(}\DataTypeTok{x=}\NormalTok{age, }\DataTypeTok{y=}\NormalTok{dep_delay_mean)) }\OperatorTok{+}
\StringTok{    }\KeywordTok{geom_point}\NormalTok{()}
\end{Highlighting}
\end{Shaded}

\begin{verbatim}
## Warning: Removed 1 rows containing missing values (geom_point).
\end{verbatim}

\includegraphics{13Ex_files/figure-latex/unnamed-chunk-8-1.pdf}

\hypertarget{section-15}{%
\subsection{4}\label{section-15}}

\begin{quote}
What weather conditions make it more likely to see a delay?
\end{quote}

\begin{Shaded}
\begin{Highlighting}[]
\NormalTok{flights }\OperatorTok\StringTok{ }
\StringTok{    }\KeywordTok{inner_join}\NormalTok{(weather, }\DataTypeTok{by =} \KeywordTok{c}\NormalTok{(}\StringTok{"origin"}\NormalTok{ =}\StringTok{ "origin"}\NormalTok{,}
                   \StringTok{"year"}\NormalTok{ =}\StringTok{ "year"}\NormalTok{,}
                   \StringTok{"month"}\NormalTok{ =}\StringTok{ "month"}\NormalTok{,}
                   \StringTok{"day"}\NormalTok{ =}\StringTok{ "day"}\NormalTok{,}
                   \StringTok{"hour"}\NormalTok{ =}\StringTok{ "hour"}
\NormalTok{                   )}
\NormalTok{    )}
\end{Highlighting}
\end{Shaded}

\begin{verbatim}
## # A tibble: 335,220 x 29
##     year month   day dep_time sched_dep_time dep_delay arr_time
##    <dbl> <dbl> <int>    <int>          <int>     <dbl>    <int>
##  1  2013     1     1      517            515         2      830
##  2  2013     1     1      533            529         4      850
##  3  2013     1     1      542            540         2      923
##  4  2013     1     1      544            545        -1     1004
##  5  2013     1     1      554            600        -6      812
##  6  2013     1     1      554            558        -4      740
##  7  2013     1     1      555            600        -5      913
##  8  2013     1     1      557            600        -3      709
##  9  2013     1     1      557            600        -3      838
## 10  2013     1     1      558            600        -2      753
## # ... with 335,210 more rows, and 22 more variables: sched_arr_time <int>,
## #   arr_delay <dbl>, carrier <chr>, flight <int>, tailnum <chr>,
## #   origin <chr>, dest <chr>, air_time <dbl>, distance <dbl>, hour <dbl>,
## #   minute <dbl>, time_hour.x <dttm>, temp <dbl>, dewp <dbl>, humid <dbl>,
## #   wind_dir <dbl>, wind_speed <dbl>, wind_gust <dbl>, precip <dbl>,
## #   pressure <dbl>, visib <dbl>, time_hour.y <dttm>
\end{verbatim}

GLM\ldots?

\hypertarget{section-16}{%
\subsection{5}\label{section-16}}

\begin{quote}
What happened on June 13, 2013? Display the spatial pattern of delays,
and then use Google to cross-reference with the weather.
\end{quote}

There were storms.

\hypertarget{section-17}{%
\section{13.5.1}\label{section-17}}

\hypertarget{section-18}{%
\subsection{1}\label{section-18}}

\begin{quote}
What does it mean for a flight to have a missing \texttt{tailnum}? What
do the tail numbers that don't have a matching record in planes have in
common? (Hint: one variable explains \textasciitilde90\% of the
problems.)
\end{quote}

\begin{Shaded}
\begin{Highlighting}[]
\NormalTok{flights }\OperatorTok
\StringTok{    }\KeywordTok{anti_join}\NormalTok{(planes, }\DataTypeTok{by =} \StringTok{"tailnum"}\NormalTok{) }\OperatorTok
\StringTok{    }\KeywordTok{group_by}\NormalTok{(carrier) }\OperatorTok
\StringTok{    }\KeywordTok{summarize}\NormalTok{(}\DataTypeTok{count=}\KeywordTok{n}\NormalTok{()) }\OperatorTok
\StringTok{    }\KeywordTok{select}\NormalTok{(carrier, count)}
\end{Highlighting}
\end{Shaded}

\begin{verbatim}
## # A tibble: 10 x 2
##    carrier count
##    <chr>   <int>
##  1 9E       1044
##  2 AA      22558
##  3 B6        830
##  4 DL        110
##  5 F9         50
##  6 FL        187
##  7 MQ      25397
##  8 UA       1693
##  9 US        699
## 10 WN         38
\end{verbatim}

\hypertarget{section-19}{%
\subsection{2}\label{section-19}}

\begin{quote}
Filter flights to only show flights with planes that have flown at least
100 flights.
\end{quote}

\begin{Shaded}
\begin{Highlighting}[]
\NormalTok{flights }\OperatorTok
\StringTok{    }\KeywordTok{group_by}\NormalTok{(tailnum) }\OperatorTok
\StringTok{    }\KeywordTok{count}\NormalTok{() }\OperatorTok
\StringTok{    }\KeywordTok{filter}\NormalTok{(n }\OperatorTok{>=}\StringTok{ }\DecValTok{100}\NormalTok{)}
\end{Highlighting}
\end{Shaded}

\begin{verbatim}
## # A tibble: 1,218 x 2
## # Groups:   tailnum [1,218]
##    tailnum     n
##    <chr>   <int>
##  1 <NA>     2512
##  2 N0EGMQ    371
##  3 N10156    153
##  4 N10575    289
##  5 N11106    129
##  6 N11107    148
##  7 N11109    148
##  8 N11113    138
##  9 N11119    148
## 10 N11121    154
## # ... with 1,208 more rows
\end{verbatim}

\hypertarget{section-20}{%
\subsection{3}\label{section-20}}

\begin{quote}
Combine \texttt{fueleconomy::vehicles} and \texttt{fueleconomy::common}
to find only the records for the most common models.
\end{quote}

\begin{Shaded}
\begin{Highlighting}[]
\NormalTok{fueleconomy}\OperatorTok{::}\NormalTok{vehicles }\OperatorTok
\StringTok{    }\KeywordTok{semi_join}\NormalTok{(fueleconomy}\OperatorTok{::}\NormalTok{common, }\DataTypeTok{by =} \KeywordTok{c}\NormalTok{(}\StringTok{"make"}\NormalTok{, }\StringTok{"model"}\NormalTok{)) }\OperatorTok
\StringTok{    }\KeywordTok{distinct}\NormalTok{(model, make) }\OperatorTok
\StringTok{    }\KeywordTok{group_by}\NormalTok{(model) }\OperatorTok
\StringTok{    }\KeywordTok{mutate}\NormalTok{(}\DataTypeTok{count=}\KeywordTok{n}\NormalTok{()) }\OperatorTok
\StringTok{    }\KeywordTok{filter}\NormalTok{(count }\OperatorTok{>}\StringTok{ }\DecValTok{1}\NormalTok{) }\OperatorTok
\StringTok{    }\KeywordTok{arrange}\NormalTok{(count)}
\end{Highlighting}
\end{Shaded}

\begin{verbatim}
## # A tibble: 8 x 3
## # Groups:   model [3]
##   model     make       count
##   <chr>     <chr>      <int>
## 1 Colt      Dodge          2
## 2 Colt      Plymouth       2
## 3 Truck 2WD Mitsubishi     3
## 4 Truck 4WD Mitsubishi     3
## 5 Truck 2WD Nissan         3
## 6 Truck 4WD Nissan         3
## 7 Truck 2WD Toyota         3
## 8 Truck 4WD Toyota         3
\end{verbatim}

\hypertarget{section-21}{%
\subsection{4}\label{section-21}}

\begin{quote}
Find the 48 hours (over the course of the whole year) that have the
worst delays. Cross-reference it with the weather data. Can you see any
patterns?
\end{quote}

\begin{Shaded}
\begin{Highlighting}[]
\NormalTok{flights }\OperatorTok
\StringTok{    }\KeywordTok{mutate}\NormalTok{(}\DataTypeTok{hour=}\NormalTok{sched_dep_time }\OperatorTok\StringTok{ }\DecValTok{100}\NormalTok{) }\OperatorTok
\StringTok{    }\KeywordTok{group_by}\NormalTok{(origin, year, month, day, hour) }\OperatorTok
\StringTok{    }\KeywordTok{summarise}\NormalTok{(}\DataTypeTok{dep_delay =} \KeywordTok{mean}\NormalTok{(dep_delay, }\DataTypeTok{na.rm =} \OtherTok{TRUE}\NormalTok{)) }\OperatorTok
\StringTok{    }\KeywordTok{arrange}\NormalTok{(}\KeywordTok{desc}\NormalTok{(dep_delay)) }\OperatorTok
\StringTok{    }\KeywordTok{slice}\NormalTok{(}\DecValTok{1}\OperatorTok{:}\DecValTok{48}\NormalTok{) }\OperatorTok
\StringTok{    }\KeywordTok{inner_join}\NormalTok{(weather, }\DataTypeTok{by=}\KeywordTok{c}\NormalTok{(}\StringTok{"origin"}\NormalTok{, }\StringTok{"year"}\NormalTok{, }\StringTok{"month"}\NormalTok{, }\StringTok{"day"}\NormalTok{, }\StringTok{"hour"}\NormalTok{)) }\OperatorTok
\StringTok{    }\KeywordTok{ggplot}\NormalTok{(}\KeywordTok{aes}\NormalTok{(}\DataTypeTok{x =}\NormalTok{ precip, }\DataTypeTok{y =}\NormalTok{ wind_speed, }\DataTypeTok{color =}\NormalTok{ temp)) }\OperatorTok{+}
\StringTok{    }\KeywordTok{geom_point}\NormalTok{()}
\end{Highlighting}
\end{Shaded}

\begin{verbatim}
## Warning: Removed 4 rows containing missing values (geom_point).
\end{verbatim}

\includegraphics{13Ex_files/figure-latex/unnamed-chunk-13-1.pdf}

\hypertarget{section-22}{%
\subsection{5}\label{section-22}}

\begin{quote}
What does
\texttt{anti\_join(flights,\ airports,\ by\ =\ c("dest"\ =\ "faa"))}
tell you? What does
\texttt{anti\_join(airports,\ flights,\ by\ =\ c("faa"\ =\ "dest"))}
tell you?
\end{quote}


\end{document}
