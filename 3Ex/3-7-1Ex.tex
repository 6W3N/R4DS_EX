\documentclass[]{article}
\usepackage{lmodern}
\usepackage{amssymb,amsmath}
\usepackage{ifxetex,ifluatex}
\usepackage{fixltx2e} % provides \textsubscript
\ifnum 0\ifxetex 1\fi\ifluatex 1\fi=0 % if pdftex
  \usepackage[T1]{fontenc}
  \usepackage[utf8]{inputenc}
\else % if luatex or xelatex
  \ifxetex
    \usepackage{mathspec}
  \else
    \usepackage{fontspec}
  \fi
  \defaultfontfeatures{Ligatures=TeX,Scale=MatchLowercase}
\fi
% use upquote if available, for straight quotes in verbatim environments
\IfFileExists{upquote.sty}{\usepackage{upquote}}{}
% use microtype if available
\IfFileExists{microtype.sty}{%
\usepackage{microtype}
\UseMicrotypeSet[protrusion]{basicmath} % disable protrusion for tt fonts
}{}
\usepackage[margin=1in]{geometry}
\usepackage{hyperref}
\hypersetup{unicode=true,
            pdftitle={r4ds Ex 3.7.1},
            pdfauthor={MW},
            pdfborder={0 0 0},
            breaklinks=true}
\urlstyle{same}  % don't use monospace font for urls
\usepackage{color}
\usepackage{fancyvrb}
\newcommand{\VerbBar}{|}
\newcommand{\VERB}{\Verb[commandchars=\\\{\}]}
\DefineVerbatimEnvironment{Highlighting}{Verbatim}{commandchars=\\\{\}}
% Add ',fontsize=\small' for more characters per line
\usepackage{framed}
\definecolor{shadecolor}{RGB}{248,248,248}
\newenvironment{Shaded}{\begin{snugshade}}{\end{snugshade}}
\newcommand{\AlertTok}[1]{\textcolor[rgb]{0.94,0.16,0.16}{#1}}
\newcommand{\AnnotationTok}[1]{\textcolor[rgb]{0.56,0.35,0.01}{\textbf{\textit{#1}}}}
\newcommand{\AttributeTok}[1]{\textcolor[rgb]{0.77,0.63,0.00}{#1}}
\newcommand{\BaseNTok}[1]{\textcolor[rgb]{0.00,0.00,0.81}{#1}}
\newcommand{\BuiltInTok}[1]{#1}
\newcommand{\CharTok}[1]{\textcolor[rgb]{0.31,0.60,0.02}{#1}}
\newcommand{\CommentTok}[1]{\textcolor[rgb]{0.56,0.35,0.01}{\textit{#1}}}
\newcommand{\CommentVarTok}[1]{\textcolor[rgb]{0.56,0.35,0.01}{\textbf{\textit{#1}}}}
\newcommand{\ConstantTok}[1]{\textcolor[rgb]{0.00,0.00,0.00}{#1}}
\newcommand{\ControlFlowTok}[1]{\textcolor[rgb]{0.13,0.29,0.53}{\textbf{#1}}}
\newcommand{\DataTypeTok}[1]{\textcolor[rgb]{0.13,0.29,0.53}{#1}}
\newcommand{\DecValTok}[1]{\textcolor[rgb]{0.00,0.00,0.81}{#1}}
\newcommand{\DocumentationTok}[1]{\textcolor[rgb]{0.56,0.35,0.01}{\textbf{\textit{#1}}}}
\newcommand{\ErrorTok}[1]{\textcolor[rgb]{0.64,0.00,0.00}{\textbf{#1}}}
\newcommand{\ExtensionTok}[1]{#1}
\newcommand{\FloatTok}[1]{\textcolor[rgb]{0.00,0.00,0.81}{#1}}
\newcommand{\FunctionTok}[1]{\textcolor[rgb]{0.00,0.00,0.00}{#1}}
\newcommand{\ImportTok}[1]{#1}
\newcommand{\InformationTok}[1]{\textcolor[rgb]{0.56,0.35,0.01}{\textbf{\textit{#1}}}}
\newcommand{\KeywordTok}[1]{\textcolor[rgb]{0.13,0.29,0.53}{\textbf{#1}}}
\newcommand{\NormalTok}[1]{#1}
\newcommand{\OperatorTok}[1]{\textcolor[rgb]{0.81,0.36,0.00}{\textbf{#1}}}
\newcommand{\OtherTok}[1]{\textcolor[rgb]{0.56,0.35,0.01}{#1}}
\newcommand{\PreprocessorTok}[1]{\textcolor[rgb]{0.56,0.35,0.01}{\textit{#1}}}
\newcommand{\RegionMarkerTok}[1]{#1}
\newcommand{\SpecialCharTok}[1]{\textcolor[rgb]{0.00,0.00,0.00}{#1}}
\newcommand{\SpecialStringTok}[1]{\textcolor[rgb]{0.31,0.60,0.02}{#1}}
\newcommand{\StringTok}[1]{\textcolor[rgb]{0.31,0.60,0.02}{#1}}
\newcommand{\VariableTok}[1]{\textcolor[rgb]{0.00,0.00,0.00}{#1}}
\newcommand{\VerbatimStringTok}[1]{\textcolor[rgb]{0.31,0.60,0.02}{#1}}
\newcommand{\WarningTok}[1]{\textcolor[rgb]{0.56,0.35,0.01}{\textbf{\textit{#1}}}}
\usepackage{graphicx,grffile}
\makeatletter
\def\maxwidth{\ifdim\Gin@nat@width>\linewidth\linewidth\else\Gin@nat@width\fi}
\def\maxheight{\ifdim\Gin@nat@height>\textheight\textheight\else\Gin@nat@height\fi}
\makeatother
% Scale images if necessary, so that they will not overflow the page
% margins by default, and it is still possible to overwrite the defaults
% using explicit options in \includegraphics[width, height, ...]{}
\setkeys{Gin}{width=\maxwidth,height=\maxheight,keepaspectratio}
\IfFileExists{parskip.sty}{%
\usepackage{parskip}
}{% else
\setlength{\parindent}{0pt}
\setlength{\parskip}{6pt plus 2pt minus 1pt}
}
\setlength{\emergencystretch}{3em}  % prevent overfull lines
\providecommand{\tightlist}{%
  \setlength{\itemsep}{0pt}\setlength{\parskip}{0pt}}
\setcounter{secnumdepth}{0}
% Redefines (sub)paragraphs to behave more like sections
\ifx\paragraph\undefined\else
\let\oldparagraph\paragraph
\renewcommand{\paragraph}[1]{\oldparagraph{#1}\mbox{}}
\fi
\ifx\subparagraph\undefined\else
\let\oldsubparagraph\subparagraph
\renewcommand{\subparagraph}[1]{\oldsubparagraph{#1}\mbox{}}
\fi

%%% Use protect on footnotes to avoid problems with footnotes in titles
\let\rmarkdownfootnote\footnote%
\def\footnote{\protect\rmarkdownfootnote}

%%% Change title format to be more compact
\usepackage{titling}

% Create subtitle command for use in maketitle
\newcommand{\subtitle}[1]{
  \posttitle{
    \begin{center}\large#1\end{center}
    }
}

\setlength{\droptitle}{-2em}

  \title{r4ds Ex 3.7.1}
    \pretitle{\vspace{\droptitle}\centering\huge}
  \posttitle{\par}
    \author{MW}
    \preauthor{\centering\large\emph}
  \postauthor{\par}
      \predate{\centering\large\emph}
  \postdate{\par}
    \date{2019/05/15}


\begin{document}
\maketitle

\hypertarget{section}{%
\section{3.7.1}\label{section}}

\hypertarget{section-1}{%
\subsection{1}\label{section-1}}

\begin{Shaded}
\begin{Highlighting}[]
\KeywordTok{ggplot}\NormalTok{(}\DataTypeTok{data=}\NormalTok{diamonds) }\OperatorTok{+}
\StringTok{    }\KeywordTok{geom_pointrange}\NormalTok{(}\KeywordTok{aes}\NormalTok{(}\DataTypeTok{x=}\NormalTok{cut, }\DataTypeTok{y=}\NormalTok{depth), }\DataTypeTok{stat=}\StringTok{"summary"}\NormalTok{)}
\end{Highlighting}
\end{Shaded}

\begin{verbatim}
## No summary function supplied, defaulting to `mean_se()
\end{verbatim}

\includegraphics{3-7-1Ex_files/figure-latex/unnamed-chunk-1-1.pdf}

\hypertarget{section-2}{%
\subsection{2}\label{section-2}}

\begin{Shaded}
\begin{Highlighting}[]
\KeywordTok{library}\NormalTok{(gridExtra)}
\NormalTok{a <-}\StringTok{ }\KeywordTok{ggplot}\NormalTok{(}\DataTypeTok{data=}\NormalTok{diamonds) }\OperatorTok{+}
\StringTok{    }\KeywordTok{geom_col}\NormalTok{(}\KeywordTok{aes}\NormalTok{(}\DataTypeTok{x=}\NormalTok{cut, }\DataTypeTok{y=}\NormalTok{depth))}

\NormalTok{b <-}\StringTok{ }\KeywordTok{ggplot}\NormalTok{(}\DataTypeTok{data=}\NormalTok{diamonds) }\OperatorTok{+}
\StringTok{    }\KeywordTok{geom_bar}\NormalTok{(}\KeywordTok{aes}\NormalTok{(}\DataTypeTok{x=}\NormalTok{cut))}

\KeywordTok{grid.arrange}\NormalTok{(a, b, }\DataTypeTok{nrow =} \DecValTok{1}\NormalTok{)}
\end{Highlighting}
\end{Shaded}

\includegraphics{3-7-1Ex_files/figure-latex/unnamed-chunk-2-1.pdf}
\texttt{geom\_col} uses \texttt{stat\_identity}, so it requires
\texttt{x} and \texttt{y} in aesthetics. \texttt{geom\_bar} uses
\texttt{stat\_count} by default, so it has possibility to change the
\texttt{y} by specifying \texttt{stat}.

Example is as follows:

\begin{Shaded}
\begin{Highlighting}[]
\KeywordTok{ggplot}\NormalTok{(}\DataTypeTok{data=}\NormalTok{diamonds) }\OperatorTok{+}
\StringTok{    }\KeywordTok{geom_bar}\NormalTok{(}\KeywordTok{aes}\NormalTok{(}\DataTypeTok{x=}\NormalTok{cut, }\DataTypeTok{y=}\NormalTok{depth), }\DataTypeTok{stat=}\StringTok{"identity"}\NormalTok{)}
\end{Highlighting}
\end{Shaded}

\includegraphics{3-7-1Ex_files/figure-latex/unnamed-chunk-3-1.pdf}

\hypertarget{section-3}{%
\subsection{3}\label{section-3}}

Correspondence list of \texttt{geom\_} and \texttt{stat\_}. -
\texttt{geom\_bar} \texttt{stat\_count} - \texttt{geom\_contour}
\texttt{stat\_contour} - \texttt{geom\_count} \texttt{stat\_sum} -
\texttt{geom\_density} \texttt{stat\_density} - \texttt{geom\_histgram}
\texttt{stat\_bin} - \texttt{geom\_qq\_line} \texttt{stat\_qq\_line} -
\texttt{geom\_qq} \texttt{stat\_qq} - \texttt{geom\_smooth}
\texttt{stat\_smooth} - \texttt{geom\_violin} \texttt{stat\_violin} -
\texttt{geom\_bin2d} \texttt{stat\_bin\_2d} - \texttt{geom\_boxplot}
\texttt{stat\_boxplot} - \texttt{geom\_density\_2d}
\texttt{stat\_density\_2d} - \texttt{geom\_hex} \texttt{stat\_hex} -
\texttt{geom\_freqpoly} \texttt{stat\_bin} - \texttt{geom\_quantile}
\texttt{stat\_quantile} - \texttt{geom\_sf} \texttt{stat\_sf}

\hypertarget{section-4}{%
\subsection{4}\label{section-4}}

\begin{Shaded}
\begin{Highlighting}[]
\NormalTok{stat_smooth}
\end{Highlighting}
\end{Shaded}

\begin{itemize}
\tightlist
\item
  \texttt{y}: predicted value
\item
  \texttt{ymin}: lower value of the confidence interval
\item
  \texttt{ymax}: upper value of the confidence interval
\item
  \texttt{se}: standard error
\end{itemize}

The list of parameters that control the behavior is as follows: -
\texttt{method} - \texttt{formula} - \texttt{se} - \texttt{na.rm} -
\texttt{show.legend} - \texttt{inherit.aes} - \texttt{n} - \texttt{span}
- \texttt{fullrange} - \texttt{level} - \texttt{method.args}

\hypertarget{section-5}{%
\subsection{5}\label{section-5}}

\begin{Shaded}
\begin{Highlighting}[]
\KeywordTok{ggplot}\NormalTok{(}\DataTypeTok{data =}\NormalTok{ diamonds) }\OperatorTok{+}\StringTok{ }
\StringTok{  }\KeywordTok{geom_bar}\NormalTok{(}\DataTypeTok{mapping =} \KeywordTok{aes}\NormalTok{(}\DataTypeTok{x =}\NormalTok{ cut, }\DataTypeTok{y =}\NormalTok{ ..prop..))}
\end{Highlighting}
\end{Shaded}

\includegraphics{3-7-1Ex_files/figure-latex/unnamed-chunk-5-1.pdf}

\begin{Shaded}
\begin{Highlighting}[]
\KeywordTok{ggplot}\NormalTok{(}\DataTypeTok{data =}\NormalTok{ diamonds) }\OperatorTok{+}\StringTok{ }
\StringTok{  }\KeywordTok{geom_bar}\NormalTok{(}\DataTypeTok{mapping =} \KeywordTok{aes}\NormalTok{(}\DataTypeTok{x =}\NormalTok{ cut, }\DataTypeTok{fill =}\NormalTok{ color, }\DataTypeTok{y =}\NormalTok{ ..prop..))}
\end{Highlighting}
\end{Shaded}

\includegraphics{3-7-1Ex_files/figure-latex/unnamed-chunk-5-2.pdf} The
problem is that the ratio calculation is done within the group.
Solutions;

\begin{Shaded}
\begin{Highlighting}[]
\KeywordTok{ggplot}\NormalTok{(}\DataTypeTok{data=}\NormalTok{diamonds) }\OperatorTok{+}\StringTok{ }
\StringTok{    }\KeywordTok{geom_bar}\NormalTok{(}\KeywordTok{aes}\NormalTok{(}\DataTypeTok{x=}\NormalTok{cut, }\DataTypeTok{y=}\NormalTok{..prop.., }\DataTypeTok{group=}\DecValTok{1}\NormalTok{))}
\end{Highlighting}
\end{Shaded}

\includegraphics{3-7-1Ex_files/figure-latex/unnamed-chunk-6-1.pdf}

\begin{Shaded}
\begin{Highlighting}[]
\NormalTok{diamonds }\OperatorTok\StringTok{ }
\StringTok{    }\KeywordTok{group_by}\NormalTok{(cut) }\OperatorTok\StringTok{ }\KeywordTok{nest}\NormalTok{() }\OperatorTok\StringTok{ }
\StringTok{    }\KeywordTok{inner_join}\NormalTok{(}\KeywordTok{count}\NormalTok{(diamonds }\OperatorTok\StringTok{ }\KeywordTok{group_by}\NormalTok{(cut))) }\OperatorTok\StringTok{ }
\StringTok{    }\KeywordTok{mutate}\NormalTok{(}\DataTypeTok{s=}\NormalTok{.}\OperatorTok{$}\NormalTok{n }\OperatorTok\StringTok{ }\NormalTok{sum) }\OperatorTok\StringTok{ }\KeywordTok{unnest}\NormalTok{() }\OperatorTok\StringTok{ }
\StringTok{    }\KeywordTok{ggplot}\NormalTok{(}\KeywordTok{aes}\NormalTok{(}\DataTypeTok{x=}\NormalTok{cut, }\DataTypeTok{y=}\NormalTok{n}\OperatorTok{/}\NormalTok{s, }\DataTypeTok{fill=}\NormalTok{color)) }\OperatorTok{+}
\StringTok{        }\KeywordTok{geom_bar}\NormalTok{(}\DataTypeTok{stat=}\StringTok{"identity"}\NormalTok{)}
\end{Highlighting}
\end{Shaded}

\begin{verbatim}
## Joining, by = "cut"
\end{verbatim}

\includegraphics{3-7-1Ex_files/figure-latex/unnamed-chunk-6-2.pdf}


\end{document}
