\documentclass[]{ltjsarticle}
\usepackage{lmodern}
\usepackage{amssymb,amsmath}
\usepackage{ifxetex,ifluatex}
\usepackage{fixltx2e} % provides \textsubscript
\ifnum 0\ifxetex 1\fi\ifluatex 1\fi=0 % if pdftex
  \usepackage[T1]{fontenc}
  \usepackage[utf8]{inputenc}
\else % if luatex or xelatex
  \ifxetex
    \usepackage{mathspec}
  \else
    \usepackage{fontspec}
  \fi
  \defaultfontfeatures{Ligatures=TeX,Scale=MatchLowercase}
\fi
% use upquote if available, for straight quotes in verbatim environments
\IfFileExists{upquote.sty}{\usepackage{upquote}}{}
% use microtype if available
\IfFileExists{microtype.sty}{%
\usepackage{microtype}
\UseMicrotypeSet[protrusion]{basicmath} % disable protrusion for tt fonts
}{}
\usepackage[margin=1in]{geometry}
\usepackage{hyperref}
\hypersetup{unicode=true,
            pdftitle={r4ds Ex 12},
            pdfauthor={MW},
            pdfborder={0 0 0},
            breaklinks=true}
\urlstyle{same}  % don't use monospace font for urls
\usepackage{color}
\usepackage{fancyvrb}
\newcommand{\VerbBar}{|}
\newcommand{\VERB}{\Verb[commandchars=\\\{\}]}
\DefineVerbatimEnvironment{Highlighting}{Verbatim}{commandchars=\\\{\}}
% Add ',fontsize=\small' for more characters per line
\usepackage{framed}
\definecolor{shadecolor}{RGB}{248,248,248}
\newenvironment{Shaded}{\begin{snugshade}}{\end{snugshade}}
\newcommand{\AlertTok}[1]{\textcolor[rgb]{0.94,0.16,0.16}{#1}}
\newcommand{\AnnotationTok}[1]{\textcolor[rgb]{0.56,0.35,0.01}{\textbf{\textit{#1}}}}
\newcommand{\AttributeTok}[1]{\textcolor[rgb]{0.77,0.63,0.00}{#1}}
\newcommand{\BaseNTok}[1]{\textcolor[rgb]{0.00,0.00,0.81}{#1}}
\newcommand{\BuiltInTok}[1]{#1}
\newcommand{\CharTok}[1]{\textcolor[rgb]{0.31,0.60,0.02}{#1}}
\newcommand{\CommentTok}[1]{\textcolor[rgb]{0.56,0.35,0.01}{\textit{#1}}}
\newcommand{\CommentVarTok}[1]{\textcolor[rgb]{0.56,0.35,0.01}{\textbf{\textit{#1}}}}
\newcommand{\ConstantTok}[1]{\textcolor[rgb]{0.00,0.00,0.00}{#1}}
\newcommand{\ControlFlowTok}[1]{\textcolor[rgb]{0.13,0.29,0.53}{\textbf{#1}}}
\newcommand{\DataTypeTok}[1]{\textcolor[rgb]{0.13,0.29,0.53}{#1}}
\newcommand{\DecValTok}[1]{\textcolor[rgb]{0.00,0.00,0.81}{#1}}
\newcommand{\DocumentationTok}[1]{\textcolor[rgb]{0.56,0.35,0.01}{\textbf{\textit{#1}}}}
\newcommand{\ErrorTok}[1]{\textcolor[rgb]{0.64,0.00,0.00}{\textbf{#1}}}
\newcommand{\ExtensionTok}[1]{#1}
\newcommand{\FloatTok}[1]{\textcolor[rgb]{0.00,0.00,0.81}{#1}}
\newcommand{\FunctionTok}[1]{\textcolor[rgb]{0.00,0.00,0.00}{#1}}
\newcommand{\ImportTok}[1]{#1}
\newcommand{\InformationTok}[1]{\textcolor[rgb]{0.56,0.35,0.01}{\textbf{\textit{#1}}}}
\newcommand{\KeywordTok}[1]{\textcolor[rgb]{0.13,0.29,0.53}{\textbf{#1}}}
\newcommand{\NormalTok}[1]{#1}
\newcommand{\OperatorTok}[1]{\textcolor[rgb]{0.81,0.36,0.00}{\textbf{#1}}}
\newcommand{\OtherTok}[1]{\textcolor[rgb]{0.56,0.35,0.01}{#1}}
\newcommand{\PreprocessorTok}[1]{\textcolor[rgb]{0.56,0.35,0.01}{\textit{#1}}}
\newcommand{\RegionMarkerTok}[1]{#1}
\newcommand{\SpecialCharTok}[1]{\textcolor[rgb]{0.00,0.00,0.00}{#1}}
\newcommand{\SpecialStringTok}[1]{\textcolor[rgb]{0.31,0.60,0.02}{#1}}
\newcommand{\StringTok}[1]{\textcolor[rgb]{0.31,0.60,0.02}{#1}}
\newcommand{\VariableTok}[1]{\textcolor[rgb]{0.00,0.00,0.00}{#1}}
\newcommand{\VerbatimStringTok}[1]{\textcolor[rgb]{0.31,0.60,0.02}{#1}}
\newcommand{\WarningTok}[1]{\textcolor[rgb]{0.56,0.35,0.01}{\textbf{\textit{#1}}}}
\usepackage{graphicx,grffile}
\makeatletter
\def\maxwidth{\ifdim\Gin@nat@width>\linewidth\linewidth\else\Gin@nat@width\fi}
\def\maxheight{\ifdim\Gin@nat@height>\textheight\textheight\else\Gin@nat@height\fi}
\makeatother
% Scale images if necessary, so that they will not overflow the page
% margins by default, and it is still possible to overwrite the defaults
% using explicit options in \includegraphics[width, height, ...]{}
\setkeys{Gin}{width=\maxwidth,height=\maxheight,keepaspectratio}
\IfFileExists{parskip.sty}{%
\usepackage{parskip}
}{% else
\setlength{\parindent}{0pt}
\setlength{\parskip}{6pt plus 2pt minus 1pt}
}
\setlength{\emergencystretch}{3em}  % prevent overfull lines
\providecommand{\tightlist}{%
  \setlength{\itemsep}{0pt}\setlength{\parskip}{0pt}}
\setcounter{secnumdepth}{0}
% Redefines (sub)paragraphs to behave more like sections
\ifx\paragraph\undefined\else
\let\oldparagraph\paragraph
\renewcommand{\paragraph}[1]{\oldparagraph{#1}\mbox{}}
\fi
\ifx\subparagraph\undefined\else
\let\oldsubparagraph\subparagraph
\renewcommand{\subparagraph}[1]{\oldsubparagraph{#1}\mbox{}}
\fi

%%% Use protect on footnotes to avoid problems with footnotes in titles
\let\rmarkdownfootnote\footnote%
\def\footnote{\protect\rmarkdownfootnote}

%%% Change title format to be more compact
\usepackage{titling}

% Create subtitle command for use in maketitle
\newcommand{\subtitle}[1]{
  \posttitle{
    \begin{center}\large#1\end{center}
    }
}

\setlength{\droptitle}{-2em}

  \title{r4ds Ex 12}
    \pretitle{\vspace{\droptitle}\centering\huge}
  \posttitle{\par}
    \author{MW}
    \preauthor{\centering\large\emph}
  \postauthor{\par}
      \predate{\centering\large\emph}
  \postdate{\par}
    \date{2019/06/25}


\begin{document}
\maketitle

\hypertarget{section}{%
\section{12.2}\label{section}}

\hypertarget{section-1}{%
\subsection{1}\label{section-1}}

\begin{quote}
Using prose, describe how the variables and observations are organized
in each of the sample tables.
\end{quote}

\begin{verbatim}
?talbe1
\end{verbatim}

\begin{quote}
\begin{quote}
`table1', `table2', `table3', `table4a', `table4b', and `table5' all
display the number of TB cases documented by the World Health
Organization in Afghanistan, Brazil, and China between 1999 and 2000.
The data contains values associated with four variables (country, year,
cases, and population), but each table organizes the values in a
different layout. The data is a subset of the data contained in the
World Health Organization Global Tuberculosis Report
\end{quote}
\end{quote}

\hypertarget{section-2}{%
\subsection{2}\label{section-2}}

\begin{quote}
Compute the rate for table2, and table4a + table4b. You will need to
perform four operations:
\end{quote}

\begin{Shaded}
\begin{Highlighting}[]
\NormalTok{table2 }\OperatorTok\StringTok{ }\KeywordTok{spread}\NormalTok{(type, count) }\OperatorTok\StringTok{ }
\StringTok{    }\KeywordTok{mutate}\NormalTok{(}\DataTypeTok{cases_per_cap =}\NormalTok{ (cases }\OperatorTok{/}\StringTok{ }\NormalTok{population) }\OperatorTok{*}\StringTok{ }\DecValTok{10000}\NormalTok{) }\OperatorTok
\StringTok{    }\KeywordTok{gather}\NormalTok{(}\DataTypeTok{key=}\NormalTok{type, }\DataTypeTok{value=}\NormalTok{count, cases, population, cases_per_cap)}
\end{Highlighting}
\end{Shaded}

\begin{verbatim}
## # A tibble: 18 x 4
##    country      year type            count
##    <chr>       <int> <chr>           <dbl>
##  1 Afghanistan  1999 cases         7.45e+2
##  2 Afghanistan  2000 cases         2.67e+3
##  3 Brazil       1999 cases         3.77e+4
##  4 Brazil       2000 cases         8.05e+4
##  5 China        1999 cases         2.12e+5
##  6 China        2000 cases         2.14e+5
##  7 Afghanistan  1999 population    2.00e+7
##  8 Afghanistan  2000 population    2.06e+7
##  9 Brazil       1999 population    1.72e+8
## 10 Brazil       2000 population    1.75e+8
## 11 China        1999 population    1.27e+9
## 12 China        2000 population    1.28e+9
## 13 Afghanistan  1999 cases_per_cap 3.73e-1
## 14 Afghanistan  2000 cases_per_cap 1.29e+0
## 15 Brazil       1999 cases_per_cap 2.19e+0
## 16 Brazil       2000 cases_per_cap 4.61e+0
## 17 China        1999 cases_per_cap 1.67e+0
## 18 China        2000 cases_per_cap 1.67e+0
\end{verbatim}

\begin{Shaded}
\begin{Highlighting}[]
\NormalTok{table4a }\OperatorTok\StringTok{ }\KeywordTok{inner_join}\NormalTok{(table4b, }\DataTypeTok{by=}\StringTok{"country"}\NormalTok{) }\OperatorTok\StringTok{ }
\StringTok{    }\KeywordTok{mutate}\NormalTok{(}\StringTok{`}\DataTypeTok{1999}\StringTok{`}\NormalTok{=}\StringTok{`}\DataTypeTok{1999.x}\StringTok{`}\OperatorTok{/}\StringTok{`}\DataTypeTok{1999.y}\StringTok{`}\OperatorTok{*}\DecValTok{10000}\NormalTok{, }
        \StringTok{`}\DataTypeTok{2000}\StringTok{`}\NormalTok{=}\StringTok{`}\DataTypeTok{2000.x}\StringTok{`}\OperatorTok{/}\StringTok{`}\DataTypeTok{2000.y}\StringTok{`}\OperatorTok{*}\DecValTok{10000}\NormalTok{) }\OperatorTok
\StringTok{    }\KeywordTok{select}\NormalTok{(country, }\StringTok{`}\DataTypeTok{1999}\StringTok{`}\NormalTok{, }\StringTok{`}\DataTypeTok{2000}\StringTok{`}\NormalTok{)}
\end{Highlighting}
\end{Shaded}

\begin{verbatim}
## # A tibble: 3 x 3
##   country     `1999` `2000`
##   <chr>        <dbl>  <dbl>
## 1 Afghanistan  0.373   1.29
## 2 Brazil       2.19    4.61
## 3 China        1.67    1.67
\end{verbatim}

I think the latter is harder than the former because the latter is
separated by default and it makes us be hard to manipulate columns.

\hypertarget{section-3}{%
\subsection{3}\label{section-3}}

\begin{quote}
Recreate the plot showing change in cases over time using table2 instead
of table1. What do you need to do first?
\end{quote}

\begin{Shaded}
\begin{Highlighting}[]
\NormalTok{table2 }\OperatorTok\StringTok{ }\KeywordTok{filter}\NormalTok{(type}\OperatorTok{==}\StringTok{"cases"}\NormalTok{) }\OperatorTok
\StringTok{    }\KeywordTok{ggplot}\NormalTok{(}\KeywordTok{aes}\NormalTok{(year, count)) }\OperatorTok{+}
\StringTok{    }\KeywordTok{geom_line}\NormalTok{(}\KeywordTok{aes}\NormalTok{(}\DataTypeTok{group=}\NormalTok{country), }\DataTypeTok{colour=}\StringTok{"grey50"}\NormalTok{) }\OperatorTok{+}
\StringTok{    }\KeywordTok{geom_point}\NormalTok{(}\KeywordTok{aes}\NormalTok{(}\DataTypeTok{colour=}\NormalTok{country)) }\OperatorTok{+}
\StringTok{    }\KeywordTok{scale_x_continuous}\NormalTok{(}\DataTypeTok{breaks=}\KeywordTok{unique}\NormalTok{(table2}\OperatorTok{$}\NormalTok{year)) }\OperatorTok{+}
\StringTok{    }\KeywordTok{ylab}\NormalTok{(}\StringTok{"cases"}\NormalTok{)}
\end{Highlighting}
\end{Shaded}

\includegraphics{12Ex_files/figure-latex/unnamed-chunk-3-1.pdf}

\hypertarget{section-4}{%
\section{12.3}\label{section-4}}

\hypertarget{section-5}{%
\subsection{1}\label{section-5}}

\begin{quote}
Why are \texttt{gather()} and \texttt{spread()} not perfectly
symmetrical? Carefully consider the following example:
\end{quote}

\texttt{gather()} can't specify \texttt{double} in \texttt{key}, on the
other hand \texttt{spread} can.

\hypertarget{section-6}{%
\subsection{2}\label{section-6}}

\begin{quote}
Why does this code fail?
\end{quote}

1999 -\textgreater{} \texttt{1999} 2000 -\textgreater{} \texttt{2000}

\hypertarget{section-7}{%
\subsection{3}\label{section-7}}

\begin{quote}
Why does spreading this tibble fail? How could you add a new column to
fix the problem?
\end{quote}

Rows containing \texttt{Phillip\ Woods} and \texttt{age} are duplicated.

\begin{Shaded}
\begin{Highlighting}[]
\NormalTok{people <-}\StringTok{ }\KeywordTok{tribble}\NormalTok{(}
  \OperatorTok{~}\NormalTok{name,             }\OperatorTok{~}\NormalTok{key,    }\OperatorTok{~}\NormalTok{value,}
  \CommentTok{#-----------------|--------|------}
  \StringTok{"Phillip Woods"}\NormalTok{,   }\StringTok{"age"}\NormalTok{,       }\DecValTok{45}\NormalTok{,}
  \StringTok{"Phillip Woods"}\NormalTok{,   }\StringTok{"height"}\NormalTok{,   }\DecValTok{186}\NormalTok{,}
  \StringTok{"Phillip Woods"}\NormalTok{,   }\StringTok{"age"}\NormalTok{,       }\DecValTok{50}\NormalTok{,}
  \StringTok{"Jessica Cordero"}\NormalTok{, }\StringTok{"age"}\NormalTok{,       }\DecValTok{37}\NormalTok{,}
  \StringTok{"Jessica Cordero"}\NormalTok{, }\StringTok{"height"}\NormalTok{,   }\DecValTok{156}
\NormalTok{)}

\NormalTok{people }\OperatorTok\StringTok{ }\KeywordTok{mutate}\NormalTok{(}\DataTypeTok{id=}\KeywordTok{rep}\NormalTok{(}\DecValTok{1}\OperatorTok{:}\KeywordTok{nrow}\NormalTok{(people))) }\OperatorTok\StringTok{ }\KeywordTok{spread}\NormalTok{(}\DataTypeTok{key=}\NormalTok{key, }\DataTypeTok{value=}\NormalTok{value)}
\end{Highlighting}
\end{Shaded}

\begin{verbatim}
## # A tibble: 5 x 4
##   name               id   age height
##   <chr>           <int> <dbl>  <dbl>
## 1 Jessica Cordero     4    37     NA
## 2 Jessica Cordero     5    NA    156
## 3 Phillip Woods       1    45     NA
## 4 Phillip Woods       2    NA    186
## 5 Phillip Woods       3    50     NA
\end{verbatim}

\hypertarget{section-8}{%
\subsection{4}\label{section-8}}

\begin{quote}
Tidy the simple tibble below. Do you need to spread or gather it? What
are the variables?
\end{quote}

\begin{Shaded}
\begin{Highlighting}[]
\NormalTok{preg <-}\StringTok{ }\KeywordTok{tribble}\NormalTok{(}
  \OperatorTok{~}\NormalTok{pregnant, }\OperatorTok{~}\NormalTok{male, }\OperatorTok{~}\NormalTok{female,}
  \StringTok{"yes"}\NormalTok{,     }\OtherTok{NA}\NormalTok{,    }\DecValTok{10}\NormalTok{,}
  \StringTok{"no"}\NormalTok{,      }\DecValTok{20}\NormalTok{,    }\DecValTok{12}
\NormalTok{)}

\NormalTok{preg }\OperatorTok\StringTok{ }\KeywordTok{gather}\NormalTok{(male, female, }\DataTypeTok{key=}\NormalTok{pregnant, }\DataTypeTok{value=}\NormalTok{sex, }\DataTypeTok{na.rm =} \OtherTok{TRUE}\NormalTok{)}
\end{Highlighting}
\end{Shaded}

\begin{verbatim}
## # A tibble: 3 x 2
##   pregnant   sex
##   <chr>    <dbl>
## 1 male        20
## 2 female      10
## 3 female      12
\end{verbatim}

\hypertarget{section-9}{%
\section{12.4.3}\label{section-9}}

\hypertarget{section-10}{%
\subsection{1}\label{section-10}}

\begin{quote}
What do the extra and fill arguments do in separate()? Experiment with
the various options for the following two toy datasets.
\end{quote}

\begin{Shaded}
\begin{Highlighting}[]
\KeywordTok{tibble}\NormalTok{(}\DataTypeTok{x =} \KeywordTok{c}\NormalTok{(}\StringTok{"a,b,c"}\NormalTok{, }\StringTok{"d,e,f,g"}\NormalTok{, }\StringTok{"h,i,j"}\NormalTok{)) }\OperatorTok\StringTok{ }
\StringTok{  }\KeywordTok{separate}\NormalTok{(x, }\KeywordTok{c}\NormalTok{(}\StringTok{"one"}\NormalTok{, }\StringTok{"two"}\NormalTok{, }\StringTok{"three"}\NormalTok{))}
\end{Highlighting}
\end{Shaded}

\begin{verbatim}
## Warning: Expected 3 pieces. Additional pieces discarded in 1 rows [2].
\end{verbatim}

\begin{verbatim}
## # A tibble: 3 x 3
##   one   two   three
##   <chr> <chr> <chr>
## 1 a     b     c    
## 2 d     e     f    
## 3 h     i     j
\end{verbatim}

\begin{Shaded}
\begin{Highlighting}[]
\KeywordTok{tibble}\NormalTok{(}\DataTypeTok{x =} \KeywordTok{c}\NormalTok{(}\StringTok{"a,b,c"}\NormalTok{, }\StringTok{"d,e"}\NormalTok{, }\StringTok{"f,g,i"}\NormalTok{)) }\OperatorTok\StringTok{ }
\StringTok{  }\KeywordTok{separate}\NormalTok{(x, }\KeywordTok{c}\NormalTok{(}\StringTok{"one"}\NormalTok{, }\StringTok{"two"}\NormalTok{, }\StringTok{"three"}\NormalTok{))}
\end{Highlighting}
\end{Shaded}

\begin{verbatim}
## Warning: Expected 3 pieces. Missing pieces filled with `NA` in 1 rows [2].
\end{verbatim}

\begin{verbatim}
## # A tibble: 3 x 3
##   one   two   three
##   <chr> <chr> <chr>
## 1 a     b     c    
## 2 d     e     <NA> 
## 3 f     g     i
\end{verbatim}

\texttt{extra:\ If\ ‘sep’\ is\ a\ character\ vector,\ this\ controls\ what\ happens\ when\ there\ are\ too\ many\ pieces.\ There\ are\ three\ valid\ options:\ -\ "warn"\ (the\ default):\ emit\ a\ warning\ and\ drop\ extra\ values.\ -\ "drop":\ drop\ any\ extra\ values\ without\ a\ warning.\ -\ "merge":\ only\ splits\ at\ most\ ‘length(into)’\ times}

\texttt{fill:\ If\ ‘sep’\ is\ a\ character\ vector,\ this\ controls\ what\ happens\ when\ there\ are\ not\ enough\ pieces.\ There\ are\ three\ valid\ options:\ -\ "warn"\ (the\ default):\ emit\ a\ warning\ and\ fill\ from\ the\ right\ -\ "right":\ fill\ with\ missing\ values\ on\ the\ right\ -\ "left":\ fill\ with\ missing\ values\ on\ the\ left}

\begin{Shaded}
\begin{Highlighting}[]
\KeywordTok{tibble}\NormalTok{(}\DataTypeTok{x =} \KeywordTok{c}\NormalTok{(}\StringTok{"a,b,c"}\NormalTok{, }\StringTok{"d,e,f,g"}\NormalTok{, }\StringTok{"h,i,j"}\NormalTok{)) }\OperatorTok
\StringTok{    }\KeywordTok{separate}\NormalTok{(x, }\KeywordTok{c}\NormalTok{(}\StringTok{"one"}\NormalTok{, }\StringTok{"two"}\NormalTok{, }\StringTok{"three"}\NormalTok{), }\DataTypeTok{extra=}\StringTok{"merge"}\NormalTok{)}
\end{Highlighting}
\end{Shaded}

\begin{verbatim}
## # A tibble: 3 x 3
##   one   two   three
##   <chr> <chr> <chr>
## 1 a     b     c    
## 2 d     e     f,g  
## 3 h     i     j
\end{verbatim}

\begin{Shaded}
\begin{Highlighting}[]
\KeywordTok{tibble}\NormalTok{(}\DataTypeTok{x =} \KeywordTok{c}\NormalTok{(}\StringTok{"a,b,c"}\NormalTok{, }\StringTok{"d,e"}\NormalTok{, }\StringTok{"f,g,i"}\NormalTok{)) }\OperatorTok
\StringTok{    }\KeywordTok{separate}\NormalTok{(x, }\KeywordTok{c}\NormalTok{(}\StringTok{"one"}\NormalTok{, }\StringTok{"two"}\NormalTok{, }\StringTok{"three"}\NormalTok{), }\DataTypeTok{fill=}\StringTok{"right"}\NormalTok{)}
\end{Highlighting}
\end{Shaded}

\begin{verbatim}
## # A tibble: 3 x 3
##   one   two   three
##   <chr> <chr> <chr>
## 1 a     b     c    
## 2 d     e     <NA> 
## 3 f     g     i
\end{verbatim}

\hypertarget{section-11}{%
\subsection{2}\label{section-11}}

\begin{quote}
Both \texttt{unite()} and \texttt{separate()} have a remove argument.
What does it do? Why would you set it to FALSE?
\end{quote}

\texttt{remove:\ If\ ‘TRUE’,\ remove\ input\ columns\ from\ output\ data\ frame.}

\hypertarget{section-12}{%
\subsection{3}\label{section-12}}

\begin{quote}
Compare and contrast \texttt{separate()} and \texttt{extract()}, Why are
there three variations of separation (by position, by separator, and
with groups), but only one unite?
\end{quote}

\texttt{separate()} can split columns into multiple columns by
separator. On the other hands, \texttt{extract()} can't. But
\texttt{extract()} can use regular expression.

\hypertarget{section-13}{%
\section{12.5.1}\label{section-13}}

\hypertarget{section-14}{%
\subsection{1}\label{section-14}}

\begin{quote}
Compare and contrast the \texttt{fill} arguments to \texttt{spread()}
and \texttt{complete()}.
\end{quote}

\begin{itemize}
\item
  \texttt{spread()}
\item
  \begin{itemize}
  \tightlist
  \item
    fill: If set, missing values will be replaced with this value. Note
    that there are two types of missingness in the input: explicit
    missing values (i.e.~`NA'), and implicit missings, rows that simply
    aren't present. Both types of missing value will be replaced by
    `fill'.
  \end{itemize}
\item
  \texttt{complete()}
\item
  \begin{itemize}
  \tightlist
  \item
    fill: A named list that for each variable supplies a single value to
    use instead of `NA' for missing combinations.
  \end{itemize}
\end{itemize}

\hypertarget{section-15}{%
\subsection{2}\label{section-15}}

\begin{quote}
What does the direction argument to \texttt{fill()} do?
\end{quote}

\texttt{down} and \texttt{up}.

\hypertarget{section-16}{%
\section{12.6.1}\label{section-16}}

\hypertarget{section-17}{%
\subsection{1}\label{section-17}}

\begin{quote}
In this case study, I set \texttt{na.rm\ =\ TRUE} just to make it easier
to check that we had the correct values. Is this reasonable? Think about
how missing values are represented in this dataset. Are there implicit
missing values? What's the difference between an \texttt{NA} and zero?
\end{quote}

It depends on whether \texttt{NA} in this data shows
\texttt{no\ data\ about\ TB} or
\texttt{patients\ don\textquotesingle{}t\ have\ TB}

\begin{Shaded}
\begin{Highlighting}[]
\NormalTok{who }\OperatorTok\StringTok{ }\KeywordTok{complete}\NormalTok{(year, country) }\OperatorTok\StringTok{ }\KeywordTok{nrow}\NormalTok{()}
\end{Highlighting}
\end{Shaded}

\begin{verbatim}
## [1] 7446
\end{verbatim}

\begin{Shaded}
\begin{Highlighting}[]
\NormalTok{who }\OperatorTok\StringTok{ }\KeywordTok{nrow}\NormalTok{()}
\end{Highlighting}
\end{Shaded}

\begin{verbatim}
## [1] 7240
\end{verbatim}

There are implicit rows.

\hypertarget{section-18}{%
\subsection{2}\label{section-18}}

\begin{quote}
What happens if you neglect the \texttt{mutate()} step?
(\texttt{mutate(key\ =\ stringr::str\_replace(key,\ "newrel",\ "new\_rel"))})
\end{quote}

\begin{Shaded}
\begin{Highlighting}[]
\NormalTok{who }\OperatorTok\StringTok{ }\KeywordTok{gather}\NormalTok{(key, value, new_sp_m014}\OperatorTok{:}\NormalTok{newrel_f65, }\DataTypeTok{na.rm =} \OtherTok{TRUE}\NormalTok{) }\OperatorTok
\StringTok{    }\KeywordTok{separate}\NormalTok{(key, }\KeywordTok{c}\NormalTok{(}\StringTok{"new"}\NormalTok{, }\StringTok{"var"}\NormalTok{, }\StringTok{"sexage"}\NormalTok{)) }\OperatorTok
\StringTok{    }\KeywordTok{select}\NormalTok{(}\OperatorTok{-}\NormalTok{new, }\OperatorTok{-}\NormalTok{iso2, }\OperatorTok{-}\NormalTok{iso3) }\OperatorTok
\StringTok{    }\KeywordTok{separate}\NormalTok{(sexage, }\KeywordTok{c}\NormalTok{(}\StringTok{"sex"}\NormalTok{, }\StringTok{"age"}\NormalTok{), }\DataTypeTok{sep =} \DecValTok{1}\NormalTok{)}
\end{Highlighting}
\end{Shaded}

\begin{verbatim}
## Warning: Expected 3 pieces. Missing pieces filled with `NA` in 2580 rows
## [73467, 73468, 73469, 73470, 73471, 73472, 73473, 73474, 73475, 73476,
## 73477, 73478, 73479, 73480, 73481, 73482, 73483, 73484, 73485, 73486, ...].
\end{verbatim}

\begin{verbatim}
## # A tibble: 76,046 x 6
##    country      year var   sex   age   value
##    <chr>       <int> <chr> <chr> <chr> <int>
##  1 Afghanistan  1997 sp    m     014       0
##  2 Afghanistan  1998 sp    m     014      30
##  3 Afghanistan  1999 sp    m     014       8
##  4 Afghanistan  2000 sp    m     014      52
##  5 Afghanistan  2001 sp    m     014     129
##  6 Afghanistan  2002 sp    m     014      90
##  7 Afghanistan  2003 sp    m     014     127
##  8 Afghanistan  2004 sp    m     014     139
##  9 Afghanistan  2005 sp    m     014     151
## 10 Afghanistan  2006 sp    m     014     193
## # ... with 76,036 more rows
\end{verbatim}

Many errors happen, because \texttt{separate()} emits ``too few
values''.

\hypertarget{section-19}{%
\subsection{3}\label{section-19}}

\begin{quote}
I claimed that \texttt{iso2} and \texttt{iso3} were redundant with
country. Confirm this claim.
\end{quote}

\begin{Shaded}
\begin{Highlighting}[]
\NormalTok{who }\OperatorTok\StringTok{ }\KeywordTok{nest}\NormalTok{(}\OperatorTok{-}\NormalTok{country) }\OperatorTok\StringTok{ }\KeywordTok{nrow}\NormalTok{()}
\end{Highlighting}
\end{Shaded}

\begin{verbatim}
## [1] 219
\end{verbatim}

\begin{Shaded}
\begin{Highlighting}[]
\NormalTok{who }\OperatorTok\StringTok{ }\KeywordTok{nest}\NormalTok{(}\OperatorTok{-}\NormalTok{country, }\OperatorTok{-}\NormalTok{iso2, }\OperatorTok{-}\NormalTok{iso3) }\OperatorTok\StringTok{ }\KeywordTok{nrow}\NormalTok{()}
\end{Highlighting}
\end{Shaded}

\begin{verbatim}
## [1] 219
\end{verbatim}

Above results show that \texttt{country}, \texttt{iso2}, and
\texttt{iso3} is complete matching.

\hypertarget{section-20}{%
\subsection{4}\label{section-20}}

\begin{quote}
For each country, year, and sex compute the total number of cases of TB.
Make an informative visualisation of the data.
\end{quote}

\begin{Shaded}
\begin{Highlighting}[]
\NormalTok{who }\OperatorTok\StringTok{ }\KeywordTok{gather}\NormalTok{(key, value, new_sp_m014}\OperatorTok{:}\NormalTok{newrel_f65, }\DataTypeTok{na.rm =} \OtherTok{TRUE}\NormalTok{) }\OperatorTok\StringTok{ }
\StringTok{    }\KeywordTok{mutate}\NormalTok{(}\DataTypeTok{key =}\NormalTok{ stringr}\OperatorTok{::}\KeywordTok{str_replace}\NormalTok{(key, }\StringTok{"newrel"}\NormalTok{, }\StringTok{"new_rel"}\NormalTok{)) }\OperatorTok
\StringTok{    }\KeywordTok{separate}\NormalTok{(key, }\KeywordTok{c}\NormalTok{(}\StringTok{"new"}\NormalTok{, }\StringTok{"var"}\NormalTok{, }\StringTok{"sexage"}\NormalTok{)) }\OperatorTok\StringTok{ }
\StringTok{    }\KeywordTok{select}\NormalTok{(}\OperatorTok{-}\NormalTok{new, }\OperatorTok{-}\NormalTok{iso2, }\OperatorTok{-}\NormalTok{iso3) }\OperatorTok\StringTok{ }
\StringTok{    }\KeywordTok{separate}\NormalTok{(sexage, }\KeywordTok{c}\NormalTok{(}\StringTok{"sex"}\NormalTok{, }\StringTok{"age"}\NormalTok{), }\DataTypeTok{sep =} \DecValTok{1}\NormalTok{) }\OperatorTok
\StringTok{    }\KeywordTok{group_by}\NormalTok{(country, year, sex) }\OperatorTok
\StringTok{    }\KeywordTok{summarize}\NormalTok{(}\DataTypeTok{cases=}\KeywordTok{sum}\NormalTok{(value)) }\OperatorTok
\StringTok{    }\KeywordTok{unite}\NormalTok{(country_sex, country, sex, }\DataTypeTok{remove=}\OtherTok{FALSE}\NormalTok{) }\OperatorTok
\StringTok{    }\KeywordTok{ggplot}\NormalTok{(}\KeywordTok{aes}\NormalTok{(}\DataTypeTok{x=}\NormalTok{year, }\DataTypeTok{y=}\NormalTok{cases, }\DataTypeTok{group=}\NormalTok{country_sex, }\DataTypeTok{color=}\NormalTok{sex)) }\OperatorTok{+}
\StringTok{    }\KeywordTok{geom_line}\NormalTok{()}
\end{Highlighting}
\end{Shaded}

\includegraphics{12Ex_files/figure-latex/unnamed-chunk-11-1.pdf}


\end{document}
